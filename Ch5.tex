% Ch5.tex

\chapter[Compensation for Control Packet Dropout]{Compensation for Control Packet Dropout in Networked Control Systems}
\label{cha:dropout}


Like time-varying network induced delay that have been investigated in the last two chapters, data packet dropout is another challenging problem in real-time networked control systems. Applying the real-time queuing protocol that we developed in the last chapter, we are able to limit the sum of the network induced communication delay and the control computation delay to within a control period. This one-period delay is further guaranteed by improved {\bf integrated design of real-time networked control} through embedding packet dropout compensation into the queuing protocol. 

This chapter proposes to compensate for the control packet dropout at the actuator using past control signals. Three model-free strategies for control packet dropout compensation, namely, PD (proportional plus derivative), PD2 (Proportional plus up to the second-order derivative), and PD3 (proportional plus up to the third-order derivative) are developed. They are suitable for a large number of NCS without the need to tune the compensator parameters. The proposed dropout compensation schemes are demonstrated through numerical examples.  

The core content of this work has been published in two papers \cite{TianINS07,TianIJCM07}.


%==================
\section[Philosophy for Packet Dropout Compensation]{Philosophy for Control Packet Dropout Compensation}

A queuing protocol has been developed in the last chapter for NCSs. Control packet dropout refers to the situation where, in a control period, no control packet is received by the actuator before the latest time instant to enqueue a control packet to queue~$Q_1$. 

From the timelines in Figure (...... omitted ......) for the queuing protocol, it is seen that $T_L$ is the latest {\it possible} time instant to receive control packets without packet dropout. After this time instant, some control packets may not be delivered successfully. $T_L$ is also the cut-off, i.e., latest {\em allowable}, time instant for $Q_1$ to accept control packets in a control period. Even if some control packets can reach the actuator after this time instant, they are purposely dropped. 

Before developing strategies to deal with control packet dropout, we would like to clarify the following two aspects, which are crucial for our development:
\begin{enumerate}
\parskip=0\baselineskip
\item[1)] How much computing power and other resources are available for calculations of packet dropout compensation? and
\item[2)] How much chance will various packet dropout scenarios likely occur? 
\end{enumerate}

...... More discussions are omitted here ......


%==================
\section[Mechanism for Packet Dropout Compensation]{Mechanism for Control Packet Dropout Compensation}

For real-time networked control, having the mechanism to output control signals at a fixed time instant in the real-time queuing protocol leads to more predictive timing behaviour of the NCS. This compensates for time-varying network delay, but does not solve the problem of control packet dropout. What will happen if a control packet is not received by the actuator by the time instant $T_L$? In this case, $Q_1$ is empty and no control signal can be output to the plant. We need to ``make" a control signal! This is what we will investigate in detail in the next few sections.

...... More discussions are omitted here ......